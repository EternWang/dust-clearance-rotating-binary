\documentclass[11pt]{article}
\usepackage[margin=1in]{geometry}
\usepackage{graphicx}
\usepackage{amsmath}
\usepackage{siunitx}
\usepackage{booktabs}
\usepackage{hyperref}
\usepackage{caption}
\usepackage{float}
\usepackage{setspace}
\setstretch{1.05}

\title{\textbf{Dust Clearance in a Rotating Binary}:\\
\textbf{Test-Particle Simulation of Resonance-Driven Gaps}}
\author{Hongyu Wang\\Department of Physics, University of California, Santa Barbara\\Instructor: David Berenstein}
\date{June 4, 2024}

\begin{document}
\maketitle

\begin{abstract}
Mean-motion resonances in multi-body gravitational systems can destabilize orbits and carve ``gaps'' in a particulate disk (e.g., Kirkwood gaps in the asteroid belt).
We simulate a simplified rotating-binary system (two primaries on a fixed circular orbit) and integrate thousands of test particles (``dust'') to identify which initial orbital radii are stable versus cleared.
Using a fourth-order Runge--Kutta (RK4) integrator in nondimensional units ($G=1$, $M_1=1$, $M_2=3$, separation $D=1$), we evolve particles for $t\approx\SI{40}{(time\ units)}$ and classify escape by a radial threshold.
We find a non-uniform survival probability versus initial radius, consistent with resonance-driven clearing.
\end{abstract}

\section{Model}
We consider two primaries of masses $M_1$ and $M_2$ separated by a fixed distance $D$ and moving on circular orbits about the center of mass.
We work in nondimensional units with gravitational constant $G=1$.
The binary's angular speed is
\begin{equation}
\omega = \sqrt{\frac{G(M_1+M_2)}{D^3}}.
\end{equation}
At time $t$, the primary positions are
\begin{align}
\mathbf{s}_1(t) &= -\frac{M_2}{M_1+M_2}D\,[\cos(\omega t),\,\sin(\omega t)],\\
\mathbf{s}_2(t) &= +\frac{M_1}{M_1+M_2}D\,[\cos(\omega t),\,\sin(\omega t)].
\end{align}
A test particle at position $\mathbf{r}(t)$ obeys
\begin{equation}
\ddot{\mathbf{r}} = G\,M_1\,\frac{\mathbf{s}_1(t)-\mathbf{r}}{|\mathbf{s}_1(t)-\mathbf{r}|^3} + G\,M_2\,\frac{\mathbf{s}_2(t)-\mathbf{r}}{|\mathbf{s}_2(t)-\mathbf{r}|^3}.
\end{equation}
To avoid numerical singularities during close approaches, we use a small distance ``softening'' $\epsilon$ by clamping $|\mathbf{s}-\mathbf{r}|\ge\epsilon$.

\section{Numerical methods}
\subsection{Initial conditions}
We draw $N$ particles with initial radius $r_0\sim\mathcal{U}(r_{\min}, r_{\max})$ and random orbital phase.
Initial velocities are set to an approximate Keplerian speed about the total mass $M_1+M_2$,
\begin{equation}
 v_{\mathrm{kep}}(r_0)=\sqrt{\frac{G(M_1+M_2)}{r_0}},
\end{equation}
with small Gaussian noise to break perfect symmetry.

\subsection{Integrator and escape criterion}
We integrate all particles with a fixed-step RK4 scheme using step size $\Delta t$ for a total of $N_{\mathrm{steps}}$ steps.
A particle is marked as ``escaped'' if its distance from the center of mass exceeds a threshold $r>r_{\max,\mathrm{esc}}$ at any step.
All simulation parameters (including random seed) are recorded in the output file for reproducibility.

\section{Results}
Figure~\ref{fig:survival} shows the survival fraction as a function of initial orbital radius.
A smooth gravitational potential would yield a relatively flat survival probability; in contrast, we observe structure (dips) that is consistent with resonant destabilization.

\begin{figure}[H]
  \centering
  \includegraphics[width=0.92\linewidth]{../figures/survival_fraction_vs_initial_radius.png}
  \caption{Fraction of particles surviving (not escaping beyond a threshold radius) versus initial orbital radius.}
  \label{fig:survival}
\end{figure}

Figure~\ref{fig:rot} plots final particle positions in the co-rotating frame, where the primaries are approximately stationary on the $x$-axis.
Persistent structure in this frame indicates long-lived orbital families, while depleted regions indicate clearing.

\begin{figure}[H]
  \centering
  \includegraphics[width=0.85\linewidth]{../figures/final_positions_rotating_frame.png}
  \caption{Final dust positions in the rotating frame (survivors only). Crosses mark the primary masses at their $t=0$ positions.}
  \label{fig:rot}
\end{figure}

\section{Discussion}
This simplified model reproduces a key qualitative phenomenon: gravitational resonances can preferentially remove particles from certain radii, producing gaps and overdensities.
In a more realistic disk, additional physics---collisions, gas drag, radiation pressure, and a spectrum of particle sizes---would modify both the clearing timescale and the detailed structure.
Numerically, fixed-step RK4 is adequate for qualitative behavior over modest integration times, but high-eccentricity encounters can benefit from adaptive timesteps or symplectic integrators.

\section{Reproducibility}
All figures in this report are reproducible from the code in \texttt{src/}.
A minimal workflow is:
\begin{verbatim}
python src/simulate_dust.py --n_particles 2000 --steps 2000 --dt 0.02 --seed 42
python src/make_figures.py
\end{verbatim}
See \texttt{README.md} for details.

\end{document}
